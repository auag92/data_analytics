%\begin{refsection}


\chapter{The Orbit of Mars} 
\label{kepler}


Describe the background and history briefly: who was Brahe, when what observations did he make, what
did Kepler set out to do with these observations?


\section*{The Movement of Celestrial Objects}

Describe how an observer perceives that objects move in the sky, on a daily basis and 
against the background of the fixed stars. Describe the zodiac and how the sun moves against the
background of the zodiac. Talk of the rising/setting trick. 
Describe the ecliptic. Describe the analemma and the non-uniform
motion of the sun against the zodiac. 

\section*{Sun or Earth as Center}

Describe the observations on Venus which are consistent with Sun as center.
Then describe the regressions of Mars and show that a sun-centered universe
can explain these regressions, both mathematically and pictorially. For instance,
here is the picture. Introduce the notion of oppositions. Also describe the 
angle of the Mars-Sun plane relative to the ecliptic and show how Kepler determined
this angle.

\myfig{Mars-Regression.png}{Regressions of Mars explained.}{regressions}{H}{kepler}

\section*{Brahe's Observations at Opposition}

Describe how Kepler eliminated the Earth via oppositions. Describe Brahe's longitude data
and the notion of true longitude, measured on the ecliptic place. Describe how the the period of
revolution of Mars was determined and how this can be used to calculate the mean longitude.

\section*{Modeling the Orbit of Mars II}

Describe the assumptions: circular orbit, average sun etc. Then set up the first regression problem.

\section*{Brahe's Observations with Mars in the Same Place}

Describe Brahe's data pairs with Mars in the same place but Earth at different places. 
Again set up the regression problem with this data.

\section*{Modeling the Orbit of Mars II}

Describe the assumptions: circular orbit, average sun etc. Then set up the second regression problem.

\section*{The Linear Regression Algorithm}

Outline the algorithm for linear regression.

\section*{The Non-Linear Regression Algorithm}

Outline the algorithm for the non-linear case.

\section*{Results of Regression}

Discuss the results of the regression.

\section*{Modeling the Orbit of Mars III}

Describe Kepler's new model with the ellipse.

\section*{Results of Regression with an Elliptical Model}

Discuss the results of the regression.

\section*{Conclusion}

Summarize the final results and any learning that you want to emphasize.

\FloatBarrier

%\renewcommand\bibname{{REFERENCES}} %  will print "REFERENCES" instead of "BIBLIOGRAPHY"
%\addcontentsline{toc}{chapter}{REFERENCES} %  adds "REFERENCES" to the table of content
%\printbibliography[heading=subbibliography]
%\end{refsection}








